\documentclass{article}      % Определяем класс (тип) документа
\usepackage[left=25mm, top=20mm, right=20mm, bottom=20mm, nohead, nofoot]{geometry}
\usepackage{graphicx}	% Вставка картинок
\graphicspath{{pics}}
\DeclareGraphicsExtensions{.pdf, .png, .jpg}
\usepackage[utf8]{inputenc}
\usepackage[T1]{fontenc}
\usepackage[english,russian]{babel}
\usepackage{indentfirst}
\usepackage{amsmath}
\usepackage{mathtools}
\usepackage{caption}
\usepackage{amsfonts}
\usepackage{amssymb}
\pagenumbering{gobble}
\usepackage{verbatim}
\usepackage{siunitx}

\nonfrenchspacing % Разрешаем увеличивать пробел после конца предложения

% Keywords command
\providecommand{\keywords}[1]
{
	\small
	\textbf{\textit{Keywords---}} #1
}

\title{Using an IMU Array for cycle slip detection and repair}  % Определяем заголовок
\author{Artem Novichkov$^{1}$, Ilya Goncharov$^{1}$ \\
	\small $^{1}$Bauman Moscow State Technical University
}  % Определяем имя автора
\date{13 September 2021}     % Если убрать эту команду, будет напечатана
 
\begin {document}

\maketitle                   % Выводит заголовок

\tableofcontents

% Аннотация
\begin{abstract}
	Текст аннотации
\end{abstract}

\section {Введение}


Режимы PPP и RTK являются одними из наиболее точных режимов позиционирования на сегодняшний день и позволяют  
определять позицию с миллиметровой точностью путем обработки фазовых измерений спутниковой навигационной системы. 
RTK нашел широкое практическое применение в различных областях: геодезия, сельское хозяйство, строительство. Не смотря на это, задача 
позиционирования в условиях прерывания слежения за фазой радионавигационного сигнала до сих пор является актуальной и ее 
решение востребовано на практике, особенно в условиях городской застройки. 


В данной работе рассматривается бесплатформенная инерциальная навигационная система (БИНС) на базе кластерного блока чувствительных элементов из четырех микромеханических инерциальных блоков MPU6050, проводится анализ выходных погрешностей, рассматривается вариант интеграции БИНС со спутниковым навигационным приемником для поиска и исправления скачков фазовых измерений.  
 

\section {IMU Cluster}
В настоящее время широкое распространение получили блоки инерциальных чувствительных элементов (БЧЭ) на базе MEMS-технологии.
Их основными преимуществами являются: низкая стоимость, малые массогабаритные характеристики и низкое энергопотребление.
Однако подобные приборы обладают низкой точностью: нестабильность нуля у гироскопов единицы-десятки градусов в час, а у акселерометров сотые-десятые доли милли-g.

Одним из путей повышения точности микромеханических БЧЭ является объединение их в массив (кластер). [ссылки на работы по кластеру].
Кластерные бчэ позволяют уменьшить шум в информационном сигнале в $\sqrt{N}$ раз, где $N$ - количество отдельных используемых бчэ. [ссылка]

\newpage

\begin{figure}[h!]
	\centering
	\includegraphics[width=0.3\linewidth]{cluster_pcb.jpg}
	\caption{Печатная плата с кластерным БЧЭ}
	\label{fig:cluster_pcb}
\end{figure}

Для проверки и исследования погрешностей кластерного БЧЭ разработана печатная плата (Рис. \ref{fig:cluster_pcb}).

В состав разработанного кластерного БЧЭ входят шестиосные датчики фирмы Invensense MPU6050 (трехосный акселерометр и трехосный датчик угловых скоростей).
Данные БЧЭ были выбраны вследствие их наибольшей распространнености и доступности.
Каждый из MPU6050 располагается на печатной плате в вершинах квадрата со стороной 10 мм.
Кроме того, они установлены так, чтобы соответствующие оси чувстительности каждого датчика были параллельны между собой.

Для записи данных с чувствительных элементов в составе MPU6050 были использованы следующие характеристики: диапазон измеряемых угловых скоростей $\pm$500 \SI[per-mode=symbol]{}{\degree\per\second};
диапазон измеряемых линейных ускорений $\pm$4g. Съем информации с датчиков производился на частоте 100 Гц.
Данные параметры были выбраны в предположении установки разрабатываемого устройства на маломаневренных, наземных объектах.

%Оценка смещения нуля у используемых датчиков и их дрейфа от запуска к запуску
Для калибровки и оценки смещения нуля и его дрейфа от запуска к запуску были проведены записи данных в шести положениях (Pos. 1 - Z$\uparrow$, Pos.2 - Z$\downarrow$,
Pos. 3 - X$\uparrow$, Pos.4 - X$\downarrow$, Pos. 5 - Y$\uparrow$, Pos.6 - Y$\downarrow$). Для каждого из положений производилось по три запуска, по итогам которых рассчитывались
оцениваемые параметры. Результаты измерений для акселерометров по соответствующим осям одного из БЧЭ сведены в таблицу (\ref{table:acc_rtr}) (для остальных БЧЭ полученные величины отличались незначительно).

\begin{table}[h!]
	\centering
	\caption{Смещение нуля и дрейф от запуска к запуску акселерометров MPU6050}
	\begin{tabular}{|c|c|c|c|}
	\hline
	Position & Axis & Mean, g & RTR, mg \\ \hline
	\multirow{3}{*}{Pos.1}	& X & 0.0686 & 0.068 \\											 
							\cline{2-4}
							& Y & -0.0250 & 0.06 \\												 
							\cline{2-4}
							& Z & 1.0199 & 0.2413 \\												 
	\hline
	\multirow{3}{*}{Pos.2}	& X & 0.0541 & 0.0943 \\												 
							\cline{2-4}
							& Y & -0.0086 & 0.0501 \\												 
							\cline{2-4}
							& Z & -1.0121 & 0.2652 \\												 
	\hline	
	\multirow{3}{*}{Pos.3}	& X & 1.0567 & 0.1836 \\												 
							\cline{2-4}
							& Y & -0.0246 & 0.0878 \\												 
							\cline{2-4}
							& Z & 0.0028 & 0.1113 \\												 
	\hline
	\multirow{3}{*}{Pos.4}	& X & -0.9377 & 0.1664 \\												 
							\cline{2-4}
							& Y & -0.0045 & 0.0966 \\												 
							\cline{2-4}
							& Z & -0.00004 & 0.2716 \\												 
	\hline
	\multirow{3}{*}{Pos.5}	& X & 0.0486 & 0.0729 \\												 
							\cline{2-4}
							& Y & -1.0271 & 0.0152 \\												 
							\cline{2-4}
							& Z & -0.0071 & 0.2099 \\												 
	\hline
	\multirow{3}{*}{Pos.6}	& X & 0.0731 & 0.2276 \\												 
							\cline{2-4}
							& Y & 0.9929 & 0.1313 \\												 
							\cline{2-4}
							& Z & 0.0171 & 0.2973 \\												 
	\hline
	\end{tabular}
	\label{table:acc_rtr}
\end{table}

Аналогичные результаты измерений представлены для гироскопов MPU6050 по соответствующим осям (Таблица (\ref{table:gyr_rtr})).

\begin{table}[h!]
	\centering
	\caption{Смещение нуля и дрейф от запуска к запуску гироскопов MPU6050}
	\begin{tabular}{|c|c|c|c|}
	\hline
	Position & Axis & Mean, \SI[per-mode=symbol]{}{\degree\per\second} & RTR, \SI[per-mode=symbol]{}{\degree\per\hour} \\ \hline
	\multirow{3}{*}{Pos.1}	& X & -1.0739 & 44.8406 \\												 
							\cline{2-4}
							& Y & -1.0022 & 17.3611 \\												 
							\cline{2-4}
							& Z & 1.2711 & 22.4487 \\												 
	\hline
	\multirow{3}{*}{Pos.2}	& X & -1.0246 & 59.2236 \\												 
							\cline{2-4}
							& Y & -0.9897 & 15.8603 \\												 
							\cline{2-4}
							& Z & 1.2628 & 27.1911 \\												 
	\hline	
	\multirow{3}{*}{Pos.3}	& X & -1.0122 & 66.8934 \\												 
							\cline{2-4}
							& Y & -0.9913 & 16.7310 \\												 
							\cline{2-4}
							& Z & 1.2515 & 34.0154 \\												 
	\hline
	\multirow{3}{*}{Pos.4}	& X & -1.0254 & 17.4720 \\												 
							\cline{2-4}
							& Y & -0.9991 & 7.7416 \\												 
							\cline{2-4}
							& Z & 1.2946 & 12.8763 \\												 
	\hline
	\multirow{3}{*}{Pos.5}	& X & -1.0106 & 13.5531 \\												 
							\cline{2-4}
							& Y & -0.9849 & 3.7527 \\												 
							\cline{2-4}
							& Z & 1.2519 & 7.4237 \\												 
	\hline
	\multirow{3}{*}{Pos.6}	& X & -1.0095 & 1.7701 \\												 
							\cline{2-4}
							& Y & -0.9878 & 12.9651 \\												 
							\cline{2-4}
							& Z & 1.2655 & 19.6326 \\												 
	\hline
	\end{tabular}
	\label{table:gyr_rtr}
\end{table}

Исходя из полученных значений, можно сделать вывод, что дрейф от запуска к запуску для акселерометров (max $\approx$0.3 mg) и гироскопов (max $\approx$70 \SI[per-mode=symbol]{}{\degree\per\hour}) невелик.
Это значит, что полученные оценки смещений нуля можно использовать для калибровки БЧЭ при дальнейшей работе.

%Оценка шумовых характеристик используемых датчиков по методу вариации Аллана
Также была проведена оценка шумовых характеристик микромеханических чувствительных элементов в составе MPU6050 по методу вариации Аллана. [IEEE 1431-2004, IEEE 1293-2018]
Для этого данные были записаны в течение одного часа в Pos. 1. На рисунке \ref{fig:acc_y_allan} продемонстрированы результаты расчета девиации Аллана акселерометров по оси чувствительности Y.
График с точками в виде ромбов отражает расчет девиации Аллана для кластерного БЧЭ. Расположение данного графика ниже остальных подтверждает уменьшение шума в сигнале кластерного БЧЭ 
по сравнению с сигналом единичного БЧЭ.

%Графики девиации Аллана для акселерометров по каналу Y. Сводная таблица погрешностей акселерометров.
\begin{figure}[h!]
	\centering
	\includegraphics[width=0.7\linewidth]{acc_y_adev.pdf}
	\caption{Девиация Аллана акселерометров по оси Y}
	\label{fig:acc_y_allan}
\end{figure}

\newpage

В таблицах \ref{table:acc_bias} and \ref{table:acc_vrw} представленны численные значения оценок Bias instability and Velocity Random Walk for accelerometers.

\begin{table}[h!]
	\centering
	\begin{tabular}{| c | c | c | c | c | c |}
	\hline
	Ось & IMU\_1 & IMU\_2 & IMU\_3 & IMU\_4 & Cluster \\ \hline
	X & 0.0467 & 0.0409 & 0.0417 & 0.0417 & 0.0264 \\ \hline
	Y & 0.0278 & 0.0259 & 0.0409 & 0.0344 & 0.0225 \\ \hline
	Z & 0.0764 & 0.0704 & 0.0714 & 0.0692 & 0.0472 \\
	\hline
	\end{tabular}
	\caption{Нестабильность смещения нуля акселерометров MPU6050, mg}
	\label{table:acc_bias}
\end{table}

\begin{table}[h!]
	\centering
	\begin{tabular}{| c | c | c | c | c | c |}
	\hline
	Ось & IMU\_1 & IMU\_2 & IMU\_3 & IMU\_4 & Cluster \\ \hline
	X & 0.24014 & 0.22564 & 0.24332 & 0.24124 & 0.11715 \\ \hline
	Y & 0.22065 & 0.23029 & 0.22929 & 0.22040 & 0.11211 \\ \hline
	Z & 0.35402 & 0.35872 & 0.36701 & 0.37358 & 0.18599 \\
	\hline
	\end{tabular}
	\caption{Velocity random walk (VRW) of accelerometers MPU6050, mg/$\sqrt{Hz}$}
	\label{table:acc_vrw}
\end{table}

%Выводы по данным из таблицы
Полученные результаты подтверждают гипотезу, что шумовые характеристики сигнала для акселерометров кластерного БЧЭ в $\approx\sqrt{4} = 2$ раза меньше по сравнению с характеристиками каждого
из акселерометров по отдельности.

Аналогичные оценки шума были проведены и для гироскопов в составе кластерного БЧЭ. На рисунке \ref{fig:gyr_y_allan} представлены девиации Аллана гироскопов по оси чувствительности Y.
График девиации Аллана для сигнала, полученного путем усреднения данных по четырем сигналам гироскопов (diamond) также располагается ниже графиков для каждого из гироскопов по отдельности.

%Графики девиации Аллана для гироскопов по каналу Y. Сводная таблица погрешностей гироскопов.
\begin{figure}[h!]
	\centering
	\includegraphics[width=0.7\linewidth]{gyr_y_adev.pdf}
	\caption{Девиация Аллана гироскопов по оси Y}
	\label{fig:gyr_y_allan}
\end{figure}

В таблицах \ref{table:gyro_bias} and \ref{table:gyro_arw} собраны численные значения оценок Bias instability and Angular Random Walk for gyroscopes.

\begin{table}[h!]
	\centering
	\caption{Нестабильность смещения нуля гироскопов MPU6050, \SI[per-mode=symbol]{}{\degree\per\hour}}
	\begin{tabular}{| c | c | c | c | c | c |}
	\hline
	Ось & IMU\_1 & IMU\_2 & IMU\_3 & IMU\_4 & Cluster \\ \hline
	X & 4.644 & 3.553 & 3.780 & 4.327 & 2.416 \\ \hline
	Y & 3.272 & 3.215 & 4.370 & 5.087 & 1.685 \\ \hline
	Z & 3.740 & 2.696 & 3.503 & 2.693 & 1.580 \\
	\hline
	\end{tabular}
	\label{table:gyro_bias}
\end{table}

\begin{table}[h!]
	\centering
	\caption{Angular random walk (ARW) of gyroscopes MPU6050, \SI[per-mode=symbol]{}{\degree}/$\sqrt{hr}$}
	\begin{tabular}{| c | c | c | c | c | c |}
	\hline
	Ось & IMU\_1 & IMU\_2 & IMU\_3 & IMU\_4 & Cluster \\ \hline
	X & 0.2022 & 0.1872 & 0.2382 & 0.2010 & 0.1044 \\ \hline
	Y & 0.2028 & 0.1812 & 0.1854 & 0.1938 & 0.0954 \\ \hline
	Z & 0.1638 & 0.2106 & 0.1884 & 0.1944 & 0.0972 \\
	\hline
	\end{tabular}
	\label{table:gyro_arw}
\end{table}

%Выводы по данным таблицы
Полученные характеристики для кластерного сигнала гироскопов также в $\approx2$ раза меньше величин для каждого из гироскопов по отдельности.

Таким образом, изготовленный макет подтвердил свойства кластерных БЧЭ, позволяя улучшить параметры, используемых MPU6050 в $\approx2$ раза.

\newpage

\section {Оценка выходных погрешностей}

Для оценки точности автономной работы БИНС на базе кластерного блока чувствительных элементов были использованы уравнения ошибок автономной работы ИНС. Данные уравнения учитывают медленно изменяющуюся составляющую ошибки, не зависящую от горизонтального ускорения объекта. Нестационарные погрешности, зависящие от ускорения и обусловленные погрешностью масштабных коэффициентов акселерометров, представляют собой высокочастотную ошибку, модулирующую медленно изменяющуюся шулеровскую, не учитываются в данной модели. 


Для связи выходных параметров ИНС (крен, тангаж, курс, широта, долгота) использованы следующие зависимости:  


\begin{equation}
	\label{eq:phi_x}
	\begin{gathered}
		\varPhi_x(t) = \varPhi_x(0) \cos \nu t - 
		U \cos \phi \frac {\sin \nu t} {\nu} \varPhi_z(0) - 
		\frac{\sin \nu t}{ \nu R} \delta V_y(0) +
		\frac{\sin \nu t}{\nu} \xi_x -\\ 
		U \cos \varPhi \frac{1-\cos \nu t}{\nu^2}\xi_z - 
		\frac{1-\cos \nu t}{\nu^2R} B_y(0)
	\end{gathered}
\end{equation}


\begin{equation}
	\label{eq:phi_y}
		\begin{gathered}
		\varPhi_y(t) = \varPhi_y(0) \cos \nu t + 
		\frac {\sin \nu t} {\nu R} \delta V_x(0) +
		\frac{\sin \nu t}{\nu} \xi_y +\\  
		\frac{1 - \cos \nu t}{ \nu^2 R} B_y(0)
	\end{gathered}
\end{equation}


\begin{equation}
	\label{eq:phi_z}
	\begin{gathered}
		\varPhi_z(t) = \varPhi_x(0) U \cos \phi t + 
		\varPhi_y(0) \tg \phi + \frac{ \tg \phi \sin \nu t }{ \nu R } \delta V_x(0) - \\
		(t - \frac{\sin \nu t}{\nu}) \tg \phi \xi_y + \tg \phi \frac{ 1-\cos \nu t }{ \nu^2 R } B_x(0) + \\
		\varPhi_z(0) + \xi_z t
	\end{gathered}
\end{equation}


\begin{equation}
	\label{eq:V_e}
	\begin{gathered}
		\delta V_e(t) = - \varPhi_y(0) R \sin \nu t + \delta V_x(0) \cos \nu t - \xi_y R (1 - \cos \nu t) + \\
		\frac{\sin \ nu t}{ \nu } B_x(0)
	\end{gathered}
\end{equation}


\begin{equation}
	\label{eq:V_n}
	\begin{gathered}
		\delta V_n(t) = - \varPhi_x(0) R \sin \nu t + \delta V_y(0) \cos \nu t + \xi_x R (1 - \cos \nu t) + \\
		\frac{\sin \ nu t}{ \nu } B_y(0)
	\end{gathered}
\end{equation}

\begin{equation}
	\label{eq:lambda}
	\begin{gathered}
		\lambda (t) = \int \frac{\delta V_e(t)}{R \cos \phi} dt
	\end{gathered}
\end{equation}

\begin{equation}
	\label{eq:phi}
	\begin{gathered}
		\phi (t) = \int \frac{\delta V_n(t)}{R \cos \phi} dt
	\end{gathered}
\end{equation}


\vspace{0.5cm}
где { \large $ \nu = \sqrt{ \frac{g}{R} } $ } - шулеровская частота колебания;


\vspace{0.5cm}
{\large $ B_x(0), B_y(0) $} - смещения нулей акселерометра;


\vspace{0.5cm}
{\large $ \xi_x, \xi_y, \xi_z $} - дрейф гироскопа;


\vspace{0.5cm}
Для оценки остаточной случайной составляющей погрешности акселерометра и гироскопа, с целью использования
данных значений в уравнениях  ~(\ref{eq:phi_x}) --- ~(\ref{eq:phi}) проведена запись показаний чувствиетльных элементов на протяжении 2-ух часов. По результатам измерений построены графики девиации Аллана. 

\newpage

\begin{figure}[h!]
	\centering
	\includegraphics[width=0.9\linewidth]{acc_av_adev.pdf}
	\caption{График девиации Аллана}
	\label{fig:accAdev}
\end{figure}


Полученный график Рис.\ref{fig:accAdev} девиации Аллана позволяет оценить остаточную случайную погрешность, которая будет присутствовать в выходных данных показаний акселерометров в составе кластерного инерциального измерительного блока. Исходя из уравнений ~(\ref{eq:phi_x}) -- ~(\ref{eq:phi}) остаточная погрешность акселерометра не вносит существенный вклад в ошибку определения выходных координат. Данная ошибка будет влиять исключительно на расчет пространственной ориентации.  


Ниже представлены графики нарастания ошибки определения угла тангажа и крена в течение 5 секунд автономной работы кластерной инерциальной навигационной системы. 

\newpage

\begin{figure}[h!]
	\centering
	\includegraphics[width=0.7\linewidth]{pitch.png}
	\caption{Нарастание ошибки в определении тангажа}
	\label{fig:pitch}
	
	\vspace*{\floatsep}
	
	\includegraphics[width=0.7\linewidth]{roll.png}
	\caption{Нарастание ошибки в определении крена}
	\label{fig:roll}
	
\end{figure}

\newpage
Дрейф гироскопов является основной составляющей ошибки определения координат в процессе автономной работы ИНС. Именно поэтому основной задачей исследования точности чувствительных элементов был анализ остаточных погрешностей гироскопов. Из графика (Рис.\ref{fig:mpr}) видно, что значение дрейфа на интервале осреднения 0.01 с. не превышает 0.03375 \SI[per-mode=symbol]{}{\degree\per\second}


\begin{figure}[h!]
	\centering
	\includegraphics[width=0.9\linewidth]{gyr_av_adev.pdf}
	\caption{График девиации Аллана}
	\label{fig:mpr}
\end{figure}

\newpage

\begin{figure}[h!]
	
	\centering
	\includegraphics[width=0.8\linewidth]{longitudeError.png}
	\caption{График нарастания ошибки в определении долготы места}
	\label{fig:long_error}

	\vspace*{\floatsep}
	
	\includegraphics[width=0.8\linewidth]{latitudeError.png}
	\caption{График нарастания ошибки в определении широты места}
	\label{fig:lat_error}
\end{figure}
\newpage


\section {Интеграция ИНС/ГНСС}
Описываем интеграцию IMU и GNSS с целью фиксации cycle-slip. Это простая модель, я возьму модель для GNSS RTK, которая у нас в алгоритмах. Для ИНС по Салычеву.  
Слепим синтез. Если успеем прокатиться на неделе, то шик-блеск --- будет раздел с результатами. Если нет --- скажем эксперимент в процессе. 

\section {Сравнение кластера и аналогичного по характеристикам IMU}
Я считаю, что было бы круто сравнить данный кластер со схожим по характеристикам IMU, так как я в abstract писал, что у нас COST-EFFECTIVE решение.

\begin{thebibliography}{00}
	\bibitem{b2} Author1 Article1
\end{thebibliography}


\end {document}