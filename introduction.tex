\section {Введение}


Режимы PPP и RTK являются одними из самых точных режимов позиционирования на сегодняшний день, предоставляя пользователю возможность
определять свою позицию с миллиметровой точностью путем обработки фазовых измерений спутниковой навигационной системы. 
RTK нашел широкое практическое применение в различных областях: геодезия, сельское хозяйство, строительство. Не смотря на это, задача 
позиционирования в условиях прерывания непрерывного слежения за фазой радионавигационного сигнала до сих пор является актуальной и ее 
решение востребовано на практике, особенно в условиях городской среды. 


В данной работе рассматривается схема интеграции инерциальной навигационной системы на базе кластера
из четырех микромеханических инерциальных блоков MPU6050 и двухчастотного спутникового навигационного приемника с целью детектирования и 
исправления cycle slip-ов, возникающих вследствие прерывания слежения за фазой радионавигационного сигнала. 
 