\newpage
\section {Интеграция ИНС/ГНСС}

 
 Основной целью интеграции кластерного инерциального блока с навигационным приемником является повышение
 надежности и точности выдаваемого им решения. Для обеспечения максимальной надежности интегрированного алгоритма позиционирования необходимо воспользоваться соответсвующей схемой интеграции двух систем. 
 В данной работе используется OEM плата спутникового навигационного приемника компании NTLab. Данная плата позволяет пользователю получить доступ к срезу кодовых и фазовых измерений на текущую эпоху. Таким образом, тесная схема интеграции ИНС и СНС, которая предполагает использование псевдодальностей СНС и псевдодальности ИНС, видится наиболее оптимальным вариантом. 
 
 Модель системы для расчета углов оринетации, скорости и координат ИНС: 
 
 
 \begin{equation}
 	\label{eq:att_ins}
 	\begin{gathered}
	похую
 	\end{gathered}
 \end{equation}


\begin{equation}
	\label{eq:vel_ins}
	\begin{gathered}
	похую	
	\end{gathered}
\end{equation}

\begin{equation}
	\label{eq:coord_ins}
	\begin{gathered}
	похую	
	\end{gathered}
\end{equation}
 
 